% Компилировать xelatex'ом!
\documentclass[a4paper,12pt]{article}
\usepackage[english,russian]{babel}
\usepackage{amsmath, enumerate, multicol, listings}
\usepackage{xunicode, xltxtra, xecyr}
\setmainfont{Droid Serif}
\setmonofont{Droid Sans Mono}

% Поля
\usepackage{geometry}
\geometry{left=2cm}
\geometry{right=1.5cm}
\geometry{top=1cm}
\geometry{bottom=2cm}

% Полезности
% Полезные комманды

% Оформляем заголовок лабы
\newcommand{\labtitle}[2]{
  \title{\Large{#1} \\ \large{\textbf{#2}}}
  \author{} \date{} 
  \maketitle
}

% Делаем текст жирным
%\newcommand{\b}[1]{\textbf{#1}}

% Оборачиваем в нумерованный список
\newcommand{\enum}[1]{\begin{enumerate}#1\end{enumerate}}


\begin{document}

  % Заголовок
  \labtitle{Лабораторная работа №1}{Изучение работы команд Linux для работы с файлами}
  
  % Описание
  \begin{flushleft}
    Комманды Linux для работы с файлами:
    \begin{itemize}
      \item \cmd{cd} - переход в заданную или в домашнюю папку
      \item \cmd{pwd} - вывод имени текущего каталога
      \item \cmd{cp} - копирование одного или нескольких файлов и каталогов
      \item \cmd{rm} - удаление файлов и каталогов
      \item \cmd{ls} - вывод списка файлов и подкаталогов каталога
      \item \cmd{cat} - вывод содержимого заданного файла на экран
      \item \cmd{diff} - сравнение двух файлов и вывод различий между ними
      \item \cmd{grep} - поиск заданной строки текста в файле или нескольких файлах, в том числе и с использованием регулярных выражений
      \item \cmd{mkdir} - создание папки
      \item \cmd{touch} - создание файла
      \item \cmd{mv} - перемещение или переименование файлов и каталогов
    \end{itemize}
  \end{flushleft}
  
  \begin{flushleft}
    Для выполнения лабораторной работы необходимо запустить командную строку. Сделать это можно двумя способами:
    \begin{enumerate}
      \item Нажать \cmd{Alt+F2}, в открывшемся окне набрать \cmd{"gnome-terminal"} и нажать \cmd{Enter}
      \item Где-то там выбрать все тот же гном-терминал (:
    \end{enumerate}
  \end{flushleft}
  
  \begin{flushleft}
    Для получения более подробной информации по командам можно использовать команду \cmd{man} или запускать выбранную команду с ключом \cmd{-h}, например \cmd{cp -h}
  \end{flushleft}
  
  \begin{flushleft}
    Для работы с группой файлов в командной строке можно использовать маску фильтр. Символ \texttt{"?"} в имени файла означает, что в этом месте может быть любой символ, например \texttt{"lab?.cpp"} соответствует \texttt{"lab1.cpp"} и \texttt{"lab2.cpp"}, но уже не соответствует \texttt{"lab12.cpp"}. Символ \texttt{"*"} заменяет любое количество символов, например маске \texttt{"lab*.cpp"} соответствуют все вышеназванные примеры.
  \end{flushleft}

  \newpage

  \begin{flushleft}
    \center{\Large{Практические задания}} \\[0.5em]
    \begin{enumerate}
      \item Запустите Linux
      \item Составьте справочник для вышеприведенных команд (на русском языке), расписав какие параметры для чего нужны
      \item Поработайте с этими командами
      \item Что нужно уметь:
      \begin{itemize}
        \item Запустить командную строку
        \item Вывести имя текущего каталога
        \item Перейти в домашнюю папку
        \item Вывести список файлов и папок домашней директории
        \item Создать папку \texttt{test} в домашней папке
        \item Перейти в папку \texttt{test}
        \item Скопировать файлы с расширением \texttt{txt} из \texttt{/} в \texttt{test}
        \item Отобразить скопированные файлы с информацией о размере и правах доступа
        \item Переименовать файл \texttt{wqe.txt} в \texttt{qwe.txt}
        \item Удалить файл \texttt{qwe.txt}
        \item Найти номера строк в файле \texttt{weq.txt} содержащие текст \texttt{"asd"}
        \item Заменить файлы с расширением \texttt{txt} в папке \texttt{test} из \texttt{/} с добавлением новых файлов
        \item Сравнить файл \texttt{qwe.txt} и \texttt{eqw.txt}
        \item Создать каталог \texttt{asd} в папке \texttt{test}
        \item Переместить файлы из папки \texttt{test} в \texttt{test/asd}
        \item Удалить файлы из \texttt{test/asd}
        \item Удалить каталог \texttt{asd}
        \item Скопировать все файлы и папки из \texttt{/} в \texttt{test}
        \item Удалить папку \texttt{test} включая все ее файлы и подкаталоги
      \end{itemize}
    \end{enumerate}
  \end{flushleft}

\end{document}
