% Компилировать xelatex'ом!
\documentclass[a4paper,12pt]{article}
\usepackage[english,russian]{babel}
\usepackage{amsmath, enumerate, multicol, listings}
\usepackage{xunicode, xltxtra, xecyr}
\setmainfont{Droid Serif}
\setmonofont{Droid Sans Mono}

% Поля
\usepackage{geometry}
\geometry{left=2cm}
\geometry{right=1.5cm}
\geometry{top=1cm}
\geometry{bottom=2cm}

% Полезности
% Полезные комманды

% Оформляем заголовок лабы
\newcommand{\labtitle}[2]{
  \title{\Large{#1} \\ \large{\textbf{#2}}}
  \author{} \date{} 
  \maketitle
}

% Делаем текст жирным
%\newcommand{\b}[1]{\textbf{#1}}

% Оборачиваем в нумерованный список
\newcommand{\enum}[1]{\begin{enumerate}#1\end{enumerate}}


\begin{document}

    % Заголовок
    \labtitle{Лабораторная работа №2}{Управление процессами}

    % Описание
    \begin{flushleft}
        Комманды Windows для работы с процессами:
        \begin{itemize}
            \item \cmd{schtasks} - настраивает выполнение команд по расписанию
            \item \cmd{start} - запускает определенную команду или программу в отдельном окне
            \item \cmd{tasklist} - выводит информацию о работающих процессах
            \item \cmd{taskkill} - завершает процесс
        \end{itemize}
    \end{flushleft}
  
    \begin{flushleft}
        Для выполнения лабораторной работы необходимо запустить командную строку. Сделать это можно двумя способами:
        \begin{enumerate}
            \item Пуск $\to$ Выполнить $\to$ cmd.exe
            \item Пуск $\to$ Программы $\to$ Стандартные $\to$ Командная строка
        \end{enumerate}
    \end{flushleft}
  
    \begin{flushleft}
        Для получения более подробной информации по командам можно использовать центр справки и поддержки Windows или консольную справку по команде с помощью ключа \cmd{/?}, например \cmd{schtasks /?}.
    \end{flushleft}

    \begin{flushleft}
        \center{\Large{Практические задания}} \\[0.5em]
        \begin{enumerate}
            \item Запустите Windows
            \item Составьте справочник для вышеприведенных команд (на русском языке), расписав какие параметры для чего нужны
            \item Поработайте с этими командами
            \item Что нужно уметь (в командной строке):
            \begin{itemize}
                \item Создать задание на запуск процесса в определенное время. Например, создать задание на запуск программы \cmd{Блокнот} в определенное время
                \item Отобразить список заданий на выполнение по времени
                \item Изменить программу, которая должна быть выполнена в определенное время, например, вместо программы \cmd{Блокнот} будет выполняться программа \cmd{Калькулятор}
                \item Удалить задание
                \item Запустить программу в отдельном окне в различных режимах, например, запустить программу \cmd{Блокнот} в развернутом окне и в свернутом окне, и с ожиданием его завершения
                \item Отобразить список процессов, выполняющихся на рабочей станции
                \item Отфильтровать список процессов так, чтобы в нем остались только запущенные нами программы \cmd{Блокнот} в предыдущем задании
                \item Отобразить пользователя, который запустил на выполнение эти процессы
                \item Завершить процесс по номеру PID (идентификатору процесса)
                \item Завершить процесс по имени процесса, например, завершить все процессы \cmd{Блокнот}
            \end{itemize}
        \end{enumerate}
    \end{flushleft}

\end{document}
