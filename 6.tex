% Компилировать xelatex'ом!
\documentclass[a4paper,12pt]{article}
\usepackage[english,russian]{babel}
\usepackage{amsmath, enumerate, multicol, listings}
\usepackage{xunicode, xltxtra, xecyr}
\setmainfont{Droid Serif}
\setmonofont{Droid Sans Mono}

% Поля
\usepackage{geometry}
\geometry{left=2cm}
\geometry{right=1.5cm}
\geometry{top=1cm}
\geometry{bottom=2cm}

% Полезности
% Полезные комманды

% Оформляем заголовок лабы
\newcommand{\labtitle}[2]{
  \title{\Large{#1} \\ \large{\textbf{#2}}}
  \author{} \date{} 
  \maketitle
}

% Делаем текст жирным
%\newcommand{\b}[1]{\textbf{#1}}

% Оборачиваем в нумерованный список
\newcommand{\enum}[1]{\begin{enumerate}#1\end{enumerate}}


\begin{document}

  % Заголовок
  \labtitle{Лабораторная работа №6}{Изучение работы команд Windows для работы с сетевым окружением}
  
  % Описание
  \begin{flushleft}
    Изучить базовые команды для работы с сетевым окружением.
    Список команд с их описанием:
    \begin{itemize}
     \item {\bf ping} - проверяет соединение на уровне протокола ICMP.
     \item {\bf route} - обработка таблиц сетевых маршрутов.
     \item {\bf tracert} - определяет путь до точки назначения с помощью посылки в точку назначения эхо сообщения.
     \item {\bf net time} - установка, просмотр текущего времени по серверу времени.
     \item {\bf net use} - просмотр, изменение сетевых подключений.
     \item {\bf net view} - просмотр списка общих папок и принтеров компьютера, работающего под управлением Windows XP.
     \item {\bf ipconfig} - настройка стека протокола TCP/IP.
     \item {\bf hostname} - определение имени текущего компьютера.
     \item {\bf arp} - служит для вывода и изменения записей кэша протокола ARP.
    \end{itemize}
  \end{flushleft}
  

  % Вводная
  \begin{flushleft}
    Данная лабораторная работа выполняется из командной строки. Для ее запуска необходимо выполнить следущее:
    \begin{flushleft}
      Пуск -> Выполнить -> cmd \\[0.4cm]
    \end{flushleft}
  \end{flushleft}

  \begin{flushleft}
   Для получения более подробной информации, можно использовать центр справки и поддержки Windows или консольную справку по команде с помощью ключа {\bf /?} . Например для просмотра справки по команде {\bf PING} необходимо выполнить следующую команду {\bf PING /?}. Для просмотра справки по командам {\bf NET} необходимо набрать следующую команду {\bf NET HELP 'ИМЯ КОМАНДЫ'}, например {\bf NET HELP TIME} – просмотр справки по команде {\bf NET TIME} \\[0.2cm]
  \end{flushleft}

  \begin{flushleft}
   Для авторизации на сервере 10.3.3.222 использовать логин 1 и пароль 1. Данная работа выполняется без использования виртуальной машины!
  \end{flushleft}


  \newpage

  \begin{center}
    {\bf Практические задания}
  \end{center}

  \begin{flushleft}
    \begin{enumerate} [1. ]
     \item Запустите Windows
     \item Составьте справочник для выше приведенных команд(на русском языке), расписав какие параметры для чего нужны.
     \item Поработайте с этими командами
     \item Что нужно уметь:
      \begin{enumerate} [\bf a. ]
        \item Запустить командную строку
        \item Определить наличие в сети компьютера с именем \cmd {www.sfu-kras.ru}
        \begin{flushleft}
          \cmd {ping -n 1 www.sfu-kras.ru}
        \end{flushleft}
        \item Определить наличие в сети компьютера с именем \cmd {www.sfu-kras.ru} с временем жизни пакета (TTL) 1-10
        \begin{flushleft}
          \cmd {ping -n 1 -i 6 www.sfu-kras.ru}
        \end{flushleft}
        \item Подключится к заданному серверу \cmd {10.3.3.222}
        \item Определить имя компьютера, если его адрес \cmd {10.3.3.222}
        \begin{flushleft}
         \cmd {ping -a -n 1 10.3.3.222}\\
         \cmd {nslookup 10.3.3.222}
        \end{flushleft}
        \item Отобразить список подключенных сетевых дисков
        \begin{flushleft}
         \cmd {net use}
        \end{flushleft}
        \item Подключить сетевую папку из списка доступных сервера 10.3.3.222 на свободную метку тома.
        \begin{flushleft}
         \cmd {net use L:  \textbackslash\textbackslash 10.3.3.222\textbackslash RANDOM}
        \end{flushleft}
        \item Отключить сетевой диск подключенный в предыдущем задании
        \begin{flushleft}
         \cmd {net use L: /delete}
        \end{flushleft}
        \item Отобразить текущее время на сервере 10.3.3.222
        \begin{flushleft}
         \cmd {net time \textbackslash\textbackslash 10.3.3.222}
        \end{flushleft}
        \item Установить время на текущем компьютере по серверу 10.3.3.222
        \begin{flushleft}
         \cmd {net time \textbackslash\textbackslash 10.3.3.222 /SET } \\
         Однако, при выполнении в компьютерных классах это задание может не сработать изза отсутсвия необходимых прав, в следствии чего возникнет системная ошибка
        \end{flushleft}
        \item Определить путь прохождения пакета до компьютера \cmd {www.sfu-kras.ru}
        \begin{flushleft}
         \cmd{tracert www.sfu-kras.ru}
        \end{flushleft}
        \item Отобразить таблицу маршрутизации
        \begin{flushleft}
         \cmd {route PRINT}
        \end{flushleft}
        \item Отобразить полную информацию о состоянии стека IP
        \begin{flushleft}
         \cmd {ipconfig /all}
        \end{flushleft}
        \item Обновить состояние стека IP, переполучить IP адрес от DHCP сервера
        \begin{flushleft}
         \cmd {ipconfig /renew} \\
         На выполнение этой команды, в компьютерных классах, также не хватает прав
        \end{flushleft}
        \item Отобразить список компьютеров в текущей рабочей группе
        \begin{flushleft}
         \cmd {net view}
        \end{flushleft}
        \item Отобразить список общих папок компьютера 10.3.3.222
        \begin{flushleft}
         \cmd {net view \textbackslash\textbackslash 10.3.3.222} \\[1.5cm]
        \end{flushleft}

        \end{enumerate}
    \end{enumerate}

  \end{flushleft}

  \begin{center}
   {\bf Контрольные вопросы}
  \end{center}
  \begin{flushleft}
    \begin{enumerate}
     \item Подключить сетевой диск, и назвать его вашим именем.
     \item Получить все IP адреса, которые принадлежат домену \cmd {ya.ru}
     \item Вывести список сетевых рабочих групп
     % net view /DOMAIN
     \item Подключить сетевой диск, который сохраняется после перезагрузки
      
    \end{enumerate}

  \end{flushleft}

\end{document}
