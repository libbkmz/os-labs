% Компилировать xelatex'ом!
\documentclass[a4paper,12pt]{article}
\usepackage[english,russian]{babel}
\usepackage{amsmath, enumerate, multicol, listings}
\usepackage{xunicode, xltxtra, xecyr}
\setmainfont{Droid Serif}
\setmonofont{Droid Sans Mono}

% Поля
\usepackage{geometry}
\geometry{left=2cm}
\geometry{right=1.5cm}
\geometry{top=1cm}
\geometry{bottom=2cm}

% Полезности
% Полезные комманды

% Оформляем заголовок лабы
\newcommand{\labtitle}[2]{
  \title{\Large{#1} \\ \large{\textbf{#2}}}
  \author{} \date{} 
  \maketitle
}

% Делаем текст жирным
%\newcommand{\b}[1]{\textbf{#1}}

% Оборачиваем в нумерованный список
\newcommand{\enum}[1]{\begin{enumerate}#1\end{enumerate}}


\begin{document}

  % Заголовок
  \labtitle{Лабораторная работа №6}{Изучение работы команд Windows для работы с сетевым окружением}
  
  % Описание
  \begin{flushleft}
    Изучить базовые команды для работы с сетевым окружением.
    Список команд с их описанием:
    \begin{itemize}
     \item {\bf ping} - проверяет соединение на уровне протокола. ICMP.
     \item {\bf route} - обработка таблиц сетевых маршрутов.
     \item {\bf tracert} - определяет путь до точки назначения с помощью посылки в точку назначения эхо сообщения.
     \item {\bf net time} - установка, просмотр текущего времени по серверу времени.
     \item {\bf net use} - просмотр, изменение сетевых подключений.
     \item {\bf net view} - просмотр списка общих папок и принтеров компьютера, работающего под управлением Windows XP.
     \item {\bf ipconfig} - настройка стека протокола TCP/IP.
     \item {\bf hostname} - определение имени текущего компьютера.
     \item {\bf arp} - служит для вывода и изменения записей кэша протокола ARP.
    \end{itemize}
  \end{flushleft}
  

  % Вводная
  \begin{flushleft}
    12234325343
  \end{flushleft}

\end{document}
