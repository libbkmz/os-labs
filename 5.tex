% Компилировать xelatex'ом!
\documentclass[a4paper,12pt]{article}
\usepackage[english,russian]{babel}
\usepackage{amsmath, enumerate, multicol, listings}
\usepackage{xunicode, xltxtra, xecyr}
\setmainfont{Droid Serif}
\setmonofont{Droid Sans Mono}

\lstset{extendedchars=true}

% Поля
\usepackage{geometry}
\geometry{left=2cm}
\geometry{right=1.5cm}
\geometry{top=1cm}
\geometry{bottom=2cm}

% Полезности
% Полезные комманды

% Оформляем заголовок лабы
\newcommand{\labtitle}[2]{
  \title{\Large{#1} \\ \large{\textbf{#2}}}
  \author{} \date{} 
  \maketitle
}

% Делаем текст жирным
%\newcommand{\b}[1]{\textbf{#1}}

% Оборачиваем в нумерованный список
\newcommand{\enum}[1]{\begin{enumerate}#1\end{enumerate}}


\begin{document}

  % Заголовок
  \labtitle{Лабораторная работа №5}{Взаимодействие процессов}
  
  % Описание
  \begin{flushleft}
    Разработать приложение ''Текстовый редактор'' с использованием режимов доступа к файлу:
    
   \begin{enumerate}
     \item По чтению
     \item По записи
     \item Блокировать доступ к файлу по чтению
     \item Блокировать доступ к файлу по записи
   \end{enumerate}

  Блокировку файла необходимо реализовать используюя системные блокирующие переменные семафоры.
  \\
  Одновременно запускаются несколько экземпляров разработанного приложения, и определяются возможности доступа к файлу. Найти и обосновать верную комбинацию доступа к данным файла.

  
  Для защиты лабораторной работы необходимо предоставить отчет, состоящий из следущих обязательных частей:
  \begin{itemize}
   \item Титульный лист
   \item Задание
   \item Листинг программы
   \item Результат работы программы + скриншот программы
   \item Вывод по проделанной работе
  \end{itemize}



  А также изучить следущий теоретический материал:
  \begin{itemize}
   \item Мультипрограммирование
   \item Планирование процессов и потоков
   \item Синхронизация процессов и потоков
  \end{itemize}
\end{flushleft}

  \newpage
  \begin{center}
    \Large{Примеры исходного кода} \\[0.5em]
  \end{center}

  \begin{center}
    Unix
  \end{center}
        \begin{lstlisting}[language=c,
                           breaklines=true,
                           showtabs=false,
                           showspaces=false,
                           showstringspaces=false,
                           numbers=left,
                           extendedchars=true,
                           numberstyle=\footnotesize,
                           stepnumber=1,
                           basicstyle=\ttfamily \footnotesize]
#include <stdio.h>
#include <stdlib.h>

#include <fcntl.h>           /* For O_* constants */
#include <sys/stat.h>        /* For mode constants */
#include <semaphore.h>

#include <unistd.h>          /* For sleep */

static int value;
sem_t sem;

int main()
{
    // 
    sem_t *semaphor = sem_open( "/sem" , (O_CREAT),  (S_IRWXU), 1 );

    sem_getvalue(semaphor, &value);

    sem_wait(semaphor);

    sem_trywait(semaphor);

    return 0;
}
        \end{lstlisting}



\begin{flushleft}
  \begin{itemize}
   \item В строке №14 мы создаем именованный семафор или получает сущесвующий семафор.
   \item Функция \cmd{sem\_getvalue} записывает значение семафора в переменную \cmd{value}
   \item Функция \cmd{sem\_wait} уменьшает(блокирует) переданный ей семафор. Если значение семафора больше 0, то функция возвращает управление немелдленно. Если семафор имеет нулевое значение, выплонение кода блокируется пока он станет больше 0.
   \item Функция \cmd{sem\_trywait} аналогична функции \cmd{sem\_wait} за исключением того, что если семафор имеет нулевое значение, выполнение кода не блокируется, а ошибка записывается в переменную errno.
   \item Функция \cmd{sem\_post} увеличивает(разблокирует) семафор.
  \end{itemize}
    
\end{flushleft}



  \begin{center}
    Windows
  \end{center}

        \begin{lstlisting}[language=c,
                           breaklines=true,
                           showtabs=false,
                           showspaces=false,
                           showstringspaces=false,
                           numbers=left,
                           extendedchars=true,
                           numberstyle=\footnotesize,
                           stepnumber=1,
                           basicstyle=\ttfamily \footnotesize]
#include <stdio.h>
#include <Windows.h>
#include <fcntl.h>


int main()
{
    HANDLE sem = CreateSemaphore(NULL, 1, 1, (LPCWSTR)"locker");

    WaitForSingleObject(sem, INFINITE);

    ReleaseSemaphore(sem, 1, NULL);
    return 0;
}

\end{lstlisting}

\begin{flushleft}
  \begin{itemize}
   \item В строке №8 создаем именованный семафор с начальным и максимальным значением 1.
   \item В строке №10 Мы уменьшаем значение семафора на единицу, а если его текущее значение ноль, то мы будем ожидать бесконечное(INFINITY) время, по истечении которого прекратится ожидание, либо когда какойто другой процесс увеличит семафор.
   \item Функция \cmd{ReleaseSemaphore} увеличивает значение семафора на единицу. 
  \end{itemize}

\end{flushleft}

\end{document}
