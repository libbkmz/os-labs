% Компилировать xelatex'ом!
\documentclass[a4paper,12pt]{article}
\usepackage[english,russian]{babel}
\usepackage{amsmath, enumerate, multicol, listings}
\usepackage{xunicode, xltxtra, xecyr}
\setmainfont{Droid Serif}
\setmonofont{Droid Sans Mono}

% Поля
\usepackage{geometry}
\geometry{left=2cm}
\geometry{right=1.5cm}
\geometry{top=1cm}
\geometry{bottom=2cm}

% Полезности
% Полезные комманды

% Оформляем заголовок лабы
\newcommand{\labtitle}[2]{
  \title{\Large{#1} \\ \large{\textbf{#2}}}
  \author{} \date{} 
  \maketitle
}

% Делаем текст жирным
%\newcommand{\b}[1]{\textbf{#1}}

% Оборачиваем в нумерованный список
\newcommand{\enum}[1]{\begin{enumerate}#1\end{enumerate}}


\begin{document}

  % Заголовок
  \labtitle{Лабораторная работа №5}{Взаимодействие процессов}
  
  % Описание
  \begin{flushleft}
    Разработать приложение, которое может выполнять доступ к некоторому файлу в следующих режимах:
    
   \begin{enumerate}
     \item По чтению
     \item По записи
     \item Блокировать доступ к файлу по чтению
     \item Блокировать доступ к файлу по записи
   \end{enumerate}

  Блокировку файла необходимо реализовать используюя системные блокирующие переменные семафоры.
  \\
  Одновременно запускаются несколько экземпляров разработанного приложения, и определяются возмодности доступа к файлу. Найти и обосновать верную комбинацию доступа к данным файла.

  
  Для защиты лабораторной работы необходимо предоставить отчет, состоящий из следущих обязательных частей:
  \begin{itemize}
   \item Титульный лист
   \item Задание
   \item Листинг программы
   \item Результат работы проргаммы + скриншот программы
   \item Вывод по проделанной работе
  \end{itemize}



  А также изучить следущий теоретический материал:
  \begin{itemize}
   \item Мультипрограммирование
   \item Планирование процессов и потоков
   \item Синхронизация процессов и потоков
  \end{itemize}






\end{flushleft}

\end{document}
