% Компилировать xelatex'ом!
\documentclass[a4paper,12pt]{article}
\usepackage[english,russian]{babel}
\usepackage{amsmath, enumerate, multicol, listings}
\usepackage{xunicode, xltxtra, xecyr}
\setmainfont{Droid Serif}
\setmonofont{Droid Sans Mono}

% Поля
\usepackage{geometry}
\geometry{left=2cm}
\geometry{right=1.5cm}
\geometry{top=1cm}
\geometry{bottom=2cm}

% Полезности
% Полезные комманды

% Оформляем заголовок лабы
\newcommand{\labtitle}[2]{
  \title{\Large{#1} \\ \large{\textbf{#2}}}
  \author{} \date{} 
  \maketitle
}

% Делаем текст жирным
%\newcommand{\b}[1]{\textbf{#1}}

% Оборачиваем в нумерованный список
\newcommand{\enum}[1]{\begin{enumerate}#1\end{enumerate}}


\begin{document}

  % Заголовок
  \labtitle{Лабораторная работа №7}{Изучение методов сетевого взаимодействия}
  
  % Описание
  \begin{flushleft}
    Разработать приложение, реализующие сетевое взаимодействие на основе сокетов\\[0.4cm]
  \end{flushleft}

  \begin{flushleft}
   Первое приложение - сервер, находитсяв ожидании подключения клиента и выполняет нужное действие.\\[0.4cm]
  \end{flushleft}

  \begin{flushleft}
   Второе приложение - клиент, подключается к серверу, котороый должен находится на другом компьютере, а все клиенты на одном, и выполняет одно из следующих действий(вариант лабораторной работы):
  \end{flushleft}

  \begin{enumerate}
   \item Посимвольно, в режиме on-line передает на сервер все нажатия клавишь(буквы, цифры и пробел), в приложении предусмотреть обработку клавишы "Backspace".
   \item Передает на сервер введенное многострочное сообщение c клавиатуры.
   \item Передает на сервер, выбранный пользователем файл с диска, на сервере файл сохраняется в определенный каталог и имя файла не изменяется.
   \item Клиент подключается к серверу и передает на него свое имя, в ответ он получает список подключенных к серверу клиентов, при подключении и отключении клиента. от сервера список клиентов обновляется как на сервере, так и на клиентах.\\
  \end{enumerate}

  \begin{flushleft}
   Каждый студент для выбора варианта лабораторной работы должен использовать правило описанное в четвертой лабораторной работе.
  \end{flushleft}

  \begin{flushleft}

  Для защиты лабораторной работы необходимо предоставить отчет, состоящий из следущих обязательных частей:
  \begin{itemize}
   \item Титульный лист
   \item Задание
   \item Листинг программы
   \item Результат работы проргаммы + скриншот программы
   \item Вывод по проделанной работе
  \end{itemize}



  А также изучить следущий теоретический материал:
  \begin{itemize}
   \item Что такое протокол передачи данных, сервер, клиент, сокет, домен.
   \item Методы адресации в компьютерной сети.
   \item Модель OSI
   \item Службы WINS, DNS, DHCP
  \end{itemize}
  \end{flushleft}

  


\end{document}
