% Компилировать xelatex'ом!
\documentclass[a4paper,12pt]{article}
\usepackage[english,russian]{babel}
\usepackage{amsmath, enumerate, multicol, listings}
\usepackage{xunicode, xltxtra, xecyr}
\setmainfont{Droid Serif}
\setmonofont{Droid Sans Mono}

% Поля
\usepackage{geometry}
\geometry{left=2cm}
\geometry{right=1.5cm}
\geometry{top=1cm}
\geometry{bottom=2cm}

% Полезности
% Полезные комманды

% Оформляем заголовок лабы
\newcommand{\labtitle}[2]{
  \title{\Large{#1} \\ \large{\textbf{#2}}}
  \author{} \date{} 
  \maketitle
}

% Делаем текст жирным
%\newcommand{\b}[1]{\textbf{#1}}

% Оборачиваем в нумерованный список
\newcommand{\enum}[1]{\begin{enumerate}#1\end{enumerate}}


\begin{document}

  % Заголовок
  \labtitle{Лабораторная работа №4}{Изучение способов организации файловых систем}
  
  % Описание
  \begin{flushleft}
    Разработать приложение, моделирующее работу перемещающего загрузчика. Программа считывает двоичный файл следующей структуры:
    
    \begin{center}
      \begin{tabular}{|c|c|c|c|c|c|c|c|c|}
        \hline
        N & A$_{1}$ & ... & A$_{N}$ & D$_{0}$ & D$_{1}$ & ... & D$_{K-1}$ & D$_{K}$ \\
        \hline
      \end{tabular}
    \end{center}
    
    где:
    \begin{itemize}
      \item N - количество ячеек с адресам данных
      \item K - количество ячеек с данными
      \item A$_{1}$..A$_{N}$ - ячейки с адресами данных
      \item D$_{0}$..D$_{K}$ - ячейки с данными
    \end{itemize}
    
    Необходимо изменить значение в области данных по заданным адресам на некоторую величину, вводимую с клавиатуры. В программе предусмотреть обработку ошибок. \linebreak
    
    Предусмотрено четыре варианта лабораторной работы где все ячейки:
    \begin{enumerate}
      \item восьмиразрядные
      \item шестнадцатиразрядные
      \item двадцатичетырехразрядные
      \item тридцатидвухразрядные
    \end{enumerate}
    
    Для получения своего варианта необходимо взять свой номер в списке журнала, прибавить к нему последнюю цифру номера группы, взять остаток от деления этого числа на 4 и прибавить 1. \linebreak
    
    Для защиты лабораторной работы необходимо предоставить отчет состоящий из следующих обязательных частей:
    \begin{itemize}
      \item Титульный лист
      \item Задание
      \item Листинг программы
      \item Результаты работы программы + скриншот
      \item Вывод по проделанной работе
    \end{itemize}
    
    Также необходимо изучить следующий теоретический материал:
    \begin{itemize}
      \item Способы организации и хранения информации на внешних устройствах
      \item Понятия файловой системы, файла, каталога и т.д.
      \item Способы организации файловой системы для FAT, FAT32 и NTFS
    \end{itemize}
    
  \end{flushleft}
  
\newpage

  \begin{flushleft}
    \center{\Large{Пример считывания значения двадцатичетырехразрядной ячейки из файла}} \\[0.5em]
    \begin{lstlisting}[language=c, 
                      breaklines=true, 
                      showtabs=false, 
                      showspaces=false, 
                      showstringspaces=false,
                      basicstyle=\ttfamily \footnotesize]
    
      #include <stdlib.h>
      #include <stdio.h>

      #pragma pack(push)
      #pragma pack(1)
      #define INT_MAX 0xFFFFFF
      struct int24_s { int32_t x : 24; };
      typedef struct int24_s int_t;
      #pragma pack(pop)

      #define SUCCESS 0
      #define NO_SUCH_ADDRESS -1
      #define IO_ERROR -2

      int get_value 
      (FILE* file, size_t K, size_t N, int_t address, int_t* value) 
      {

        if (address.x < 0 || address.x > K)
          return NO_SUCH_ADDRESS;

        long int pos = ftell(file);
        fseek(file, ((N + address.x + 1) * sizeof(int_t)), SEEK_SET);
        return (fread(value, sizeof(int_t), 1, file) == 1 && fseek(file, pos, SEEK_SET) == 0) ? SUCCESS : IO_ERROR;
      }
      
    \end{lstlisting}
  \end{flushleft}


\end{document}
