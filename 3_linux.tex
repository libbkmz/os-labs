% Компилировать xelatex'ом!
\documentclass[a4paper,12pt]{article}
\usepackage[english,russian]{babel}
\usepackage{amsmath, enumerate, multicol, listings}
\usepackage{xunicode, xltxtra, xecyr}
\setmainfont{Droid Serif}
\setmonofont{Droid Sans Mono}

% Поля
\usepackage{geometry}
\geometry{left=2cm}
\geometry{right=1.5cm}
\geometry{top=1cm}
\geometry{bottom=2cm}

% Полезности
% Полезные комманды

% Оформляем заголовок лабы
\newcommand{\labtitle}[2]{
  \title{\Large{#1} \\ \large{\textbf{#2}}}
  \author{} \date{} 
  \maketitle
}

% Делаем текст жирным
%\newcommand{\b}[1]{\textbf{#1}}

% Оборачиваем в нумерованный список
\newcommand{\enum}[1]{\begin{enumerate}#1\end{enumerate}}


\begin{document}

    % Заголовок
    \labtitle{Лабораторная работа №3}{Изучение методов управления памятью в оерационных системах}

    % Описание
    \begin{flushleft}
        Разработать приложение, реализующее следующие функции:
        \begin{itemize}
            \item Выделяет всю память системы
            \item Освобождает всю выделенную память
            \item Определяет количество выделенной памяти
            \item Определяет время выделения памяти системы
            \item Определяет время освобождения памяти
        \end{itemize}
    \end{flushleft}
  
    Память необходимо выделять функциями \cmd{malloc} или \cmd{calloc} блоками по N килобайт, где N - номер студента в журнале. \\

    Образец работы программы:
    \begin{flushleft}
        \begin{lstlisting}[language=c, 
                           breaklines=true, 
                           showtabs=false, 
                           showspaces=false, 
                           showstringspaces=false,
                           basicstyle=\ttfamily \footnotesize]
             Method: Calloc 
             Allocation time: 25 sec 
             Size: 2132848640 bytes 
             Release time: 98 sec

             Method: Malloc 
             Allocation time: 29 sec 
             Size: 2127646720 bytes 
             Release time: 122 sec
        \end{lstlisting}
    \end{flushleft}

    Для защиты лабораторной работы необходимо предоставить отчет состоящий из следующих обязательных частей:
    \begin{itemize}
        \item Титульный лист
        \item Задание
        \item Листинг программы
        \item Результаты работы программы по заданному образцу + скриншот, на котором видно выделение всей памяти системы
        \item Вывод по проделанной работе
    \end{itemize}
  
    Также необходимо изучить следующий теоретический материал:
    \begin{itemize}
        \item Типы адресов
        \item Классификация методов управления памятью
        \item Методы распределения памяти без использования дискового пространства
        \item Методы распределения памяти с использованием дискового пространства
        \item Своппинг
        \item Принципы кэширования данных
    \end{itemize}

\end{document}
