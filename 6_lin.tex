% Компилировать xelatex'ом!
\documentclass[a4paper,12pt]{article}
\usepackage[english,russian]{babel}
\usepackage{amsmath, enumerate, multicol, listings}
\usepackage{xunicode, xltxtra, xecyr}
\setmainfont{Droid Serif}
\setmonofont{Droid Sans Mono}

% Поля
\usepackage{geometry}
\geometry{left=2cm}
\geometry{right=1.5cm}
\geometry{top=1cm}
\geometry{bottom=2cm}

% Полезности
% Полезные комманды

% Оформляем заголовок лабы
\newcommand{\labtitle}[2]{
  \title{\Large{#1} \\ \large{\textbf{#2}}}
  \author{} \date{} 
  \maketitle
}

% Делаем текст жирным
%\newcommand{\b}[1]{\textbf{#1}}

% Оборачиваем в нумерованный список
\newcommand{\enum}[1]{\begin{enumerate}#1\end{enumerate}}


\begin{document}

  % Заголовок
  \labtitle{Лабораторная работа №6}{Изучение работы команд Linux для работы с сетевым окружением}
  
  % Описание
  \begin{flushleft}
    Изучить базовые команды для работы с сетевым окружением.
    Список команд с их описанием:
    \begin{itemize}
     \item {\bf ping} - проверяет соединение на уровне протокола ICMP.
     \item {\bf route} - обработка таблиц сетевых маршрутов.
     \item {\bf traceroute} - определяет путь до точки назначения с помощью посылки в точку назначения эхо сообщения.
     \item {\bf ntpdate} - установка текущего времени по серверу времени.
     \item {\bf smbclient} - клиент для доступа к SMB серверу
     \item {\bf smbtree} - текстовый браузер SMB сети
     \item {\bf ifconfig} - настройка стека протокола TCP/IP.
     \item {\bf host} - определение имени компьютера
     \item {\bf arp} - служит для вывода и изменения записей кэша протокола ARP.
     \item {\bf nslookup} - Определение IP-адреса по доменному имени сайта, и наоборот.
     \item {\bf mount} - монтирование файловых систем
     \item {\bf mtr} - сетевая диагностическая утилита
    \end{itemize}
  \end{flushleft}
  

  % Вводная
  \begin{flushleft}
    Данная лабораторная работа выполняется на операционной системе Xubuntu. Образ скачиваете из папки преподавателя в \cmd {C:\textbackslash Temp}. После запуска ВМ из VirtualBox открываете коммандную строку и выполняете задания.
  \end{flushleft}

  \begin{flushleft}
   Для получения подробной информации о работе команд используйте:
   \begin{itemize}
    \item \cmd {command -h}
    \item \cmd {command --help}
    \item \cmd {man command}
   \end{itemize}
  \end{flushleft}

  \begin{flushleft}
   Для авторизации на сервере 10.3.3.222 использовать логин 1 и пароль 1. \\[0.5em]
   Для авторизации в ВМ используете логин: \cmd {student} пароль: \cmd {1}. \\[0.5em]
   Для получения прав суперпользователя root необходимо перед комаой ввести \cmd {sudo}, и пароль\cmd {1}. К примеру \cmd {sudo ping sfu-kras.ru}
  \end{flushleft}


  \newpage

  \begin{center}
    {\bf Практические задания}
  \end{center}

  \begin{flushleft}
    \begin{enumerate} [1. ]
     \item Запустите Виртуальную машину с Xubuntu.
     \item Составьте справочник для выше приведенных команд(на русском языке), расписав какие параметры для чего нужны.
     \item Поработайте с этими командами
     \item Что нужно уметь:
      \begin{enumerate} [\bf a. ]
        \item Запустить командную строку
        \item Определить наличие в сети компьютера с именем \cmd {www.sfu-kras.ru}
        \begin{flushleft}
          \cmd {ping -с 1 www.sfu-kras.ru}
        \end{flushleft}
        \item Определить наличие в сети компьютера с именем \cmd {www.sfu-kras.ru} с временем жизни пакета (TTL) 1-10
        \begin{flushleft}
          \cmd {ping -c 1 -t 6 www.sfu-kras.ru}
        \end{flushleft}
        \item Подключится к заданному серверу \cmd {10.3.3.222} по SMB протоколу
        \begin{flushleft}
         \cmd {smbclient '//10.3.3.222'} , затем стандартные shell команды. ls, cd, pwd...
        \end{flushleft}
        \item Определить имя компьютера, если его адрес \cmd {10.3.3.222}
        \begin{flushleft}
         \cmd {host 10.3.3.222}\\
         \cmd {nslookup 10.3.3.222}
        \end{flushleft}
        \item Отобразить список подключенных сетевых папок
        \begin{flushleft}
         \cmd {mount | grep 'type cifs'}
        \end{flushleft}
        \item Подключить сетевую папку из списка доступных сервера 10.3.3.222
        \begin{flushleft}
         \cmd {sudo mount -t cifs -o username="guest" //10.3.3.222/pub /mnt/samba\_mount\_point}
        \end{flushleft}
        \item Отмонтировать сетевую папку подключенную в предыдущем задании
        \begin{flushleft}
         \cmd {sudo umount /mnt/samba\_mount\_point}
        \end{flushleft}
        \item Установить время на текущем компьютере по серверу 10.3.3.222
        \begin{flushleft}
         \cmd {sudo ntpdate 10.3.3.222 }
        \end{flushleft}
        \item Определить путь прохождения пакета до компьютера \cmd {www.sfu-kras.ru}
        \begin{flushleft}
         \cmd{mtr www.sfu-kras.ru}\\
         \cmd{traceroute -n www.sfu-kras.ru}\\
        \end{flushleft}
        \item Отобразить таблицу маршрутизации
        \begin{flushleft}
         \cmd {route -n}\\
         \cmd {ip route}
        \end{flushleft}
        \item Определить шлюз по умолчанию для текущего компьютера
        \begin{flushleft}
         \cmd {route -n}
        \end{flushleft}
        \item Создать запись в таблице маршрутизации со следущими параметрами
          \begin{itemize}
           \item Адрес - \cmd {192.168.66.0}
           \item Маска - \cmd {255.255.255.0}
           \item Шлюз - определяется из предыдущего пункта
           \item Метрика - 25
           \item Интерфейс - \cmd {eth0}
          \end{itemize}
        \begin{flushleft}
         \cmd {sudo route add -net 192.168.66.0 netmask 255.255.255.0 metric 25 dev eth0}
        \end{flushleft}
        \item Удалить только что созданную запись в таблице маршрутизации
        \begin{flushleft}
          \cmd {sudo route del -net 192.168.66.0 netmask 255.255.255.0}
        \end{flushleft}
        \item Отобразить полную информацию о состоянии стека IP
        \begin{flushleft}
         \cmd {ifconfig -a}
        \end{flushleft}
        \item Обновить состояние стека IP, переполучить IP адрес от DHCP сервера
        \begin{flushleft}
         \cmd {sudo /etc/init.d/networking restart} \\
         \cmd {sudo dhclient eth0 -r } \cmd {sudo dhclient eth0}
        \end{flushleft}
        \item Отобразить список компьютеров в текущей рабочей группе
        \begin{flushleft}
         \cmd {smbtree}
        \end{flushleft}
        \item Отобразить список общих папок компьютера 10.3.3.222
        \begin{flushleft}
         \cmd {smbclient -L 10.3.3.222} \\[1.5cm]
        \end{flushleft}

        \end{enumerate}
    \end{enumerate}

  \end{flushleft}

  \begin{center}
   {\bf Контрольные вопросы}
  \end{center}
  \begin{flushleft}
    \begin{enumerate}
     \item Почему не ходят пакеты до адреса \cmd {10.0.0.1}
     \item Получить все IP адреса, которые принадлежат домену \cmd {ya.ru}
     \item Разблокировать доступ к IP-адресу \cmd {10.0.0.1}
     \item Подключить сетевую папку, которая сохраняется после перезагрузки
     \item Сменить шлюз по умолчанию, на любой по усмотрению преподавателя
      
    \end{enumerate}

  \end{flushleft}

\end{document}
