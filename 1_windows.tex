% Компилировать xelatex'ом!
\documentclass[a4paper,12pt]{article}
\usepackage[english,russian]{babel}
\usepackage{amsmath, enumerate, multicol, listings}
\usepackage{xunicode, xltxtra, xecyr}
\setmainfont{Droid Serif}
\setmonofont{Droid Sans Mono}

% Поля
\usepackage{geometry}
\geometry{left=2cm}
\geometry{right=1.5cm}
\geometry{top=1cm}
\geometry{bottom=2cm}

% Полезности
% Полезные комманды

% Оформляем заголовок лабы
\newcommand{\labtitle}[2]{
  \title{\Large{#1} \\ \large{\textbf{#2}}}
  \author{} \date{} 
  \maketitle
}

% Делаем текст жирным
%\newcommand{\b}[1]{\textbf{#1}}

% Оборачиваем в нумерованный список
\newcommand{\enum}[1]{\begin{enumerate}#1\end{enumerate}}


\begin{document}

    % Заголовок
    \labtitle{Лабораторная работа №1}{Изучение работы команд Windows для работы с файлами}
  
    % Описание
    \begin{flushleft}
        Комманды Windows для работы с файлами:
        \begin{itemize}
            \item \cmd{attrib} - позволяет просматривать, устанавливать или снимать атрибуты файла или каталога, такие как "Только чтение", "Архивный", "Системный" или "Скрытый"
            \item \cmd{chdir (cd)} - вывод имени текущего каталога или переход в заданную папку
            \item \cmd{copy} - копирование одного или нескольких файлов
            \item \cmd{del} - удаление файлов и каталогов
            \item \cmd{dir} - вывод списка файлов и подкаталогов каталога
            \item \cmd{fc} - сравнение двух файлов и вывод различий между ними
            \item \cmd{find} - поиск заданной строки текста в файле или нескольких файлах
            \item \cmd{findstr} - поиск заданной строки текста в файле или нескольких файлах с использованием регулярных выражений
            \item \cmd{mkdir} - создание папки
            \item \cmd{move} - перемещение файлов
            \item \cmd{rename (ren)} - изменяет имя файла или набора файлов
            \item \cmd{replace} - заменяет файлы в одном каталоге файлами с теми же именами из другого каталога
            \item \cmd{rmdir (rd)} - удаляет каталог
            \item \cmd{tree} - представляет графисеки дерево каталогов заданного пути или диска
            \item \cmd{xcopy} - копирует файлы и каталоги, включая подкаталоги
        \end{itemize}
    \end{flushleft}
  
    \begin{flushleft}
        Для выполнения лабораторной работы необходимо запустить командную строку. Сделать это можно двумя способами:
        \begin{enumerate}
            \item Пуск $\to$ Выполнить $\to$ cmd.exe
            \item Пуск $\to$ Программы $\to$ Стандартные $\to$ Командная строка
        \end{enumerate}
    \end{flushleft}
  
    \begin{flushleft}
        Для получения более подробной информации по командам можно использовать центр справки и поддержки Windows или консольную справку по команде с помощью ключа \cmd{/?}, например \cmd{copy /?}.
    \end{flushleft}
  
    \begin{flushleft}
        Для работы с группой файлов в командной строке можно использовать маску фильтр. Символ \texttt{"?"} в имени файла означает, что в этом месте может быть любой символ, например \texttt{"lab?.cpp"} соответствует \texttt{"lab1.cpp"} и \texttt{"lab2.cpp"}, но уже не соответствует \texttt{"lab12.cpp"}. Символ \texttt{"*"} заменяет любое количество символов, например маске \texttt{"lab*.cpp"} соответствуют все вышеназванные примеры.
    \end{flushleft}

    \newpage

    \begin{flushleft}
        \center{\Large{Практические задания}} \\[0.5em]
        \begin{enumerate}
            \item Запустите Windows
            \item Составьте справочник для вышеприведенных команд (на русском языке), расписав какие параметры для чего нужны
            \item Поработайте с этими командами
            \item Что нужно уметь:
            \begin{itemize}
                \item Запустить командную строку
                \item Вывести имя текущего каталога
                \item Перейти в папку \cmd{C:\textbackslash USERS}
                \item Вывести список файлов и папок \cmd{C:\textbackslash USERS} в несколько столбцов
                \item Создать папку \texttt{test} в \cmd{C:\textbackslash USERS}
                \item Перейти в папку \texttt{test}
                \item Скопировать все файлы с расширением \cmd{TXT} из \cmd{C:\textbackslash PROGRAM FILES\textbackslash FAR} в \cmd{C:\textbackslash USERS\textbackslash TEST}
                \item Отобразить атрибуты скопированных файлов
                \item Установить атрибут "Скрытый" для файла \cmd{MACROS.TXT}
                \item Вывести список файлов содержащих атрибут "Скрытый"
                \item Удалить атрибут "Скрытый" для файла \cmd{MACROS.TXT}
                \item Переименовать файл \texttt{MACROS.TXT} в \texttt{TEST.TXT}
                \item Удалить файл \texttt{TEST.TXT}
                \item Найти номера строк в файле \texttt{LICENSE.XUSSR.TXT} содержащие текст \texttt{"xUSSR"}
                \item Заменить файлы с расширением \texttt{TXT} в \cmd{C:\textbackslash USERS\textbackslash TEST} из \cmd{C:\textbackslash PROGRAM FILES\textbackslash FAR} с добавлением новых файлов
                \item Сравнить файлы \texttt{MACROS.RUS.TXT} и \texttt{MACROS.TXT}
                \item Создать каталог \texttt{FAR} в папке \cmd{C:\textbackslash USERS\textbackslash TEST}
                \item Переместить файлы из папки \cmd{C:\textbackslash USERS\textbackslash TEST} в \cmd{C:\textbackslash USERS\textbackslash TEST\textbackslash FAR}
                \item Построить графическое дерево каталога \cmd{C:\textbackslash USERS}
                \item Удалить файлы из \cmd{C:\textbackslash USERS\textbackslash TEST\textbackslash FAR}
                \item Удалить каталог \cmd{C:\textbackslash USERS\textbackslash TEST\textbackslash FAR}
                \item Скопировать все файлы и папки из \cmd{C:\textbackslash PROGRAM FILES\textbackslash FAR} в \cmd{C:\textbackslash USERS\textbackslash TEST}
                \item Удалить папку \cmd{C:\textbackslash USERS\textbackslash TEST} включая все ее файлы и подкаталоги
            \end{itemize}
        \end{enumerate}
    \end{flushleft}

\end{document}
